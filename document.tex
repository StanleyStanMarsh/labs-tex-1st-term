\documentclass[a5paper]{article}
\usepackage[10pt]{extsizes}

\usepackage{caption}
\usepackage{floatflt}
\usepackage{tikz}
\usepackage{array}
\usepackage[T2A]{fontenc}
\usepackage[utf8]{inputenc}
\usepackage{fancyhdr}
\usepackage[russian,english]{babel}
\usepackage{enumitem}
\usepackage{amsmath}
\usepackage[left=1.75cm,right=1.75cm,top=2cm,bottom=2cm]{geometry}

\fancyhead{}
\fancyhead[lo]{
	{\bf 71 ]}
	\vskip 5pt
}
\fancyhead[co]{
	{\footnotesize {\scriptsize\S 7. НЕКОТОРЫЕ ГЕОМЕТРИЧЕСКИЕ ПРИЛОЖЕНИЯ}}
	\vskip 5pt
}
\fancyhead[ro]{
	169
	\vskip 5pt
}
\renewcommand\headrulewidth{0pt}
\pagestyle{fancy}
\fancyfoot{}
\headsep = 7pt
\setlength{\parindent}{0.6cm}

\begin{document}
	\noindent$f''(x) ${\LARGE$>$}$0$. {\it Точки перегиба суть те ее точки, при переходе через которые} $f''(x)$ {\it меняет знак.}
	
	Из этой теоремы мы путем рассуждений, аналогичных приведенным раньше рассуждениям {\small[{\bf58}]}, получаем правило нахождения точек пере\-гиба кривой: {\it чтобы найти точки перегиба кривой, надо определить те значения $x$, при которых\quad$f''(x)$\quad обращается в нуль или не\linebreak существует, и исследовать изменение знака $f''(x)$ при переходе\linebreak через эти значения $x$, пользуясь следующей таблицей:}\\[-8mm]
	\renewcommand{\arraystretch}{1.55}
	\renewcommand{\tabcolsep}{0.2cm}
	\begin{center}
	\begin{tabular}[t]{|>{\centering\arraybackslash}p{0.6cm}|>{\centering\arraybackslash}p{1.45cm}|>{\centering\arraybackslash}p{1.45cm}|>{\centering\arraybackslash}p{1.45cm}|>{\centering\arraybackslash}p{1.45cm}|}
		\hline
					& \multicolumn{2}{c|}{{\small точка перегиба}} 	& \multicolumn{2}{c|}{{\small нет точки перегиба}}\\[0.5ex] \cline{2-5}
		{\small $f''(x)$} 	& $+-$					& $-+$			& $--$					& $++$\\
		\hline
					& {\small вогн.\:вып.}			& {\small вып.\:вогн.}	& {\small выпукл.}				& {\small вогн.}\\
		\hline
	\end{tabular}
	\end{center}

	Наиболее естественное представление об искривлении кривой мы получим, если будем следить за ихменением угла $\alpha$, составляемого касательной с осью $OX$ при движении по кривой. Из двух дуг оди\-наковой длины $\Delta s$ та дуга будет более искривлена,
	\begin{floatingfigure}{27mm}
	\begin{tikzpicture}
		\draw[thick] (0.2, 0) to [out=50, in=177] (2.45, 0.9);
		\draw[thin, ->] (0.1, 0) -- (1.52, 1.2);
		\filldraw[color=black, fill=white, thick] (0.6,0.4) circle (0.035cm);
		\node at (0.75, 0.3) {\small\textit{M}};
		\draw[thin, ->] (0.4, 0.5) -- (2, 1);
		\filldraw[color=black, fill=white, thick] (1.4,0.8) circle (0.035cm);
		\node at (1.55, 0.7) {\small\textit{N}};
		\node at (1.2, 0.5) {\scriptsize\textit{$\Delta s$}};
		\draw[thin, ->] (1.4, 0.8) -- (1.33, 1.08);
		\node at (1.6, 1.05) {\scriptsize\textit{$\Delta\alpha$}};
	\end{tikzpicture}
	\captionsetup{labelformat=empty}
	\caption{Рис. 77.}
	%\parbox[b][0.05cm][t]{27mm}{}
	\end{floatingfigure}
	\noindent
	для которой касательная повернется на больший угол, т.е. для которой приращение $\Delta\alpha$ будет\linebreak больше. Эти соображения приводят нас к понятию о средней кривизне $\Delta s$ и о кривизне в данной\linebreak точке: {\it средней кривизной дуги $\Delta s$ называется\linebreak абсолютная величина отношения угла $\Delta\alpha$ между касательными  в концах этой дуги к длине $\Delta s$ дуги. Предел этого отношения при стремлении $\Delta s$ к нулю называется кривизной кри\-вой в данной точке} (рис. 77).
	
	Таким образом, для кривизны $C$ мы получаем выражение:
	\begin{center}
		$C=|\frac{d\alpha}{dS}|$.
	\end{center}
	Но $\tg\alpha$ есть первая производная $y'$, т.е.
	\begin{center}
		$\alpha=\arctg y'$,
	\end{center}
	откуда, дифференцируя по $x$ сложную функцию $\arctg y'$:
	\begin{center}
		$d\alpha=\frac{y''}{1+y'^2}dx$.
	\end{center}

	Как мы только что показали
	\begin{center}
		$ds=\pm\sqrt{1+y'^2}dx$.
	\end{center}
	Деля $d\alpha$ на $ds$, получим окончательно выражение для кривизны
	\begin{center}
		$C=\pm\frac{y''}{(1+y'^2)^{^3\slash_2}}$.
	\end{center}
	\vspace{-0.67cm}
	\hspace{10.8cm} (5)
\end{document}